%! suppress = MissingImport
%%%%%%%%%%%%%%%%%%%%%%%%%%%%%%%%%%%%%%%%%
% Developer CV
% LaTeX Template
% Version 1.0 (28/1/19)
%
% This template originates from:
% http://www.LaTeXTemplates.com
%
% Authors:
% Jan Vorisek (jan@vorisek.me)
% Based on a template by Jan Küster (info@jankuester.com)
% Modified for LaTeX Templates by Vel (vel@LaTeXTemplates.com)
%
% License:
% The MIT License (see included LICENSE file)
%
%%%%%%%%%%%%%%%%%%%%%%%%%%%%%%%%%%%%%%%%%

%----------------------------------------------------------------------------------------
%	PACKAGES AND OTHER DOCUMENT CONFIGURATIONS
%----------------------------------------------------------------------------------------

\documentclass[8pt]{developercv} % Default font size, values from 8-12pt are recommended
\usepackage{enumitem}

%----------------------------------------------------------------------------------------
\ifdefined\generated
\else
\newcommand{\generated}{manuell}
\fi

\newcommand{\linebreaksmall}{\vspace{2mm}}
\begin{document}

%----------------------------------------------------------------------------------------
%	TITLE AND CONTACT INFORMATION
%----------------------------------------------------------------------------------------

\begin{minipage}[t]{0.45\textwidth} % 45% of the page width for name
	\vspace{-\baselineskip} % Required for vertically aligning minipages
	% If your name is very short, use just one of the lines below
	% If your name is very long, reduce the font size or make the minipage wider and reduce the others proportionately
	\colorbox{black}{{\HUGE\textcolor{white}{\textbf{\MakeUppercase{Maximilian}}}}} % First name
	
	\colorbox{black}{{\HUGE\textcolor{white}{\textbf{\MakeUppercase{Link}}}}} % Last name
	
	\vspace{6pt}
	
	{\large Full-Stack-Engineer} % Career or current job title
\end{minipage}
\begin{minipage}[t]{0.275\textwidth} % 27.5% of the page width for the first row of icons
	\vspace{-\baselineskip} % Required for vertically aligning minipages
	
	% The first parameter is the FontAwesome icon name, the second is the box size and the third is the text
	% Other icons can be found by referring to fontawesome.pdf (supplied with the template) and using the word after \fa in the command for the icon you want
	\icon{MapMarker}{9}{Karlsruhe, Deutschland}\\
	\icon{Phone}{9}{+49 172 2389581}\\
	\icon{At}{9}{\color{blue}{\href{mailto:contact@mxlink.de}{contact@mxlink.de}}}\\
\end{minipage}
\begin{minipage}[t]{0.275\textwidth} % 27.5% of the page width for the second row of icons
	\vspace{-\baselineskip} % Required for vertically aligning minipages
	
	% The first parameter is the FontAwesome icon name, the second is the box size and the third is the text
	% Other icons can be found by referring to fontawesome.pdf (supplied with the template) and using the word after \fa in the command for the icon you want
	%\icon{Globe}{12}{\href{https://alyx.vance.me}{alyx.vance.me}}\\
	\icon{Github}{9}{\color{blue}{\href{https://github.com/mxlnk}{github.com/mxlnk}}}\\
\end{minipage}

\vspace{0.5cm}

%----------------------------------------------------------------------------------------
%	INTRODUCTION, SKILLS AND TECHNOLOGIES
%----------------------------------------------------------------------------------------

\cvsect{Wer bin ich?}

\begin{minipage}[t]{0.4\textwidth} % 40% of the page width for the introduction text
	\vspace{-\baselineskip} % Required for vertically aligning minipages
	
Full-Stack-Entwickler mit über 5 Jahren Erfahrung in der Entwicklung von TypeScript-basierten SaaS-Anwendungen. 
Wichtiger technischer Beitragender in zwei wachsenden Startups, End-to-End tätig über Backend-Dienste, Frontend-Anwendungen und Cloud-Infrastruktur. 

\linebreaksmall

Mein Fokus und meine Leidenschaft liegen darin, wartbare Software zu bauen, die den Bedürfnissen der Kunden entspricht.

\end{minipage}
\hfill % Whitespace between
\begin{minipage}[t]{0.5\textwidth} % 50% of the page for the skills bar chart
	\vspace{-\baselineskip} % Required for vertically aligning minipages
	\begin{barchart}{5.5}
		\baritem{Backend (Node.js, TypeScript)}{90}
		\baritem{Frontend (React, Vue.js)}{80}
		\baritem{Datenbanken}{70}
		\baritem{Cloud-Infrastruktur}{60}
		\baritem{Python}{50}
	\end{barchart}
\end{minipage}

%\begin{center}
%	\bubbles{5/Eclipse, 6/git, 4/Office, 3/Inkscape, 3/Blender}
%\end{center}

%----------------------------------------------------------------------------------------
%	EXPERIENCE
%----------------------------------------------------------------------------------------

\cvsect{Erfahrung}

\begin{entrylist}
	\entry
		{2023 -- Heute}
		{Senior Full-Stack-Engineer}
		{octomind GmbH}
		{Entwicklung von SaaS-Werkzeugen unter Einsatz von LLMs, Agenten und Playwright, um E2E-Tests automatisch zu generieren, auszuführen und zu warten. Haupttätigkeit am Frontend mit React und Material UI sowie an einem Backend in TypeScript und Node.js.
		\begin{itemize}[nosep, topsep=0pt, left=5pt, after=\vspace{6pt}]
			\item Lange laufende Jobs vom Kerndienst abgetrennt, um erhöhte Last rund um einen nahenden Launch bewältigen zu können.
			\item Live-Parsing von Playwright-Traces, wodurch Kunden die Generierung ihrer Tests live verfolgen können.
			\item Umfangreiches Refactoring der monolithischen Codebasis in Pakete zur Verbesserung der Wartbarkeit.
			\item Mehrere größere Bugfixes, z.\,B. ein Memory Leak, das die Stabilität unseres Dienstes stark beeinträchtigte.
			\item Enge Zusammenarbeit mit Kunden und Produktteams, um technische Lösungen an Geschäftsziele anzupassen.
		\end{itemize}
		\texttt{TypeScript}\slashsep\texttt{Node.js}\slashsep\texttt{React}\slashsep\texttt{LLMs und Agenten}\slashsep\texttt{PostgreSQL}\slashsep\texttt{Docker und Kubernetes}} \linebreaksmall
	\entry
		{2017 -- 2023}
		{Gründender Full-Stack-Engineer}
		{understandAI GmbH}
		{Entwicklung von SaaS-Werkzeugen zur Annotation automobiler Sensordaten, einschließlich der Orchestrierung menschlicher, KI- und algorithmischer Workloads. Hauptsächlich Arbeit an Vue.js-basierten Frontends, mehreren Backend-Mikroservices in TypeScript und der Cloud-Infrastruktur.
		\begin{itemize}[nosep, topsep=0pt, left=5pt, after=\vspace{6pt}]
			\item Leitung der Entwicklung des internen Annotationstools, das die ersten Kundenprojekte und Umsätze ermöglichte.
			\item Entwicklung wesentlicher Teile eines Orchestrierungsdienstes zur Abbildung flexibler Workflows aus menschlichen, KI- und algorithmischen Aufgaben; ermöglichte Kundenprojekte in sehr großem Maßstab mit Hunderten Annotatoren und Millionen von Annotationen.
			\item Übernahme eines herausfordernden und ins Stocken geratenen Kundenprojekts; trug zur Wende bei und sicherte den bis dahin größten Kunden.
			\item Mentoring von Entwicklerinnen und Entwicklern, Unterstützung beim Onboarding und Mitwirkung bei Einstellungsentscheidungen.
			\item Übernahme von Teamführungsverantwortung und Verantwortung wichtiger Produktbereiche in mehreren Phasen.
		\end{itemize}
		 \texttt{TypeScript}\slashsep\texttt{Vue.js}\slashsep\texttt{Node.js}\slashsep\texttt{MongoDB}\slashsep\texttt{Python}\slashsep\texttt{Docker und Kubernetes}} \linebreaksmall
	\entry
		{2015 -- 2017\\\footnotesize{Teilzeit}}
		{Werkstudent Softwareentwicklung}
		{SearchHaus GmbH}
		{Mitarbeit an einer Wissensgraph-Suchmaschine in einem 4-Personen-Startup-Umfeld, Beitrag zu einem Java-Backend-Dienst, Erstellung eines Angular.js-Frontends zur Exploration und Abfrage des Wissensgraphen.
		\linebreaksmall \\ \texttt{Java}\slashsep\texttt{Spring Boot}\slashsep\texttt{Angular.js}}
\end{entrylist}

%----------------------------------------------------------------------------------------
%	EDUCATION
%----------------------------------------------------------------------------------------

\cvsect{Ausbildung}

\begin{entrylist}
	\entry
		{2016 -- 2020}
		{Informatik, M.Sc.}
		{Karlsruher Institut für Technologie}
		{Schwerpunkt auf Vorlesungen zu Maschinellem Lernen/Künstlicher Intelligenz, insbesondere im Bereich Computer Vision. Abschlussnote: 1,2. Abschlussarbeit: \color{blue}{\href{https://ieeexplore.ieee.org/abstract/document/9627069}{paper}}.}
	\entry
		{2012 -- 2016}
		{Informatik, B.Sc.}
		{Karlsruher Institut für Technologie}
		{Abschlussnote: 1,9.}
\end{entrylist}

%----------------------------------------------------------------------------------------
%	ADDITIONAL INFORMATION
%----------------------------------------------------------------------------------------

\begin{minipage}[t]{0.3\textwidth}
	\vspace{-\baselineskip} % Required for vertically aligning minipages

	\cvsect{Sprachen}
	\newline
	\textbf{Deutsch} - Muttersprache\\
	\textbf{Englisch} - fließend\\


\end{minipage}
\hfill
\begin{minipage}[t]{0.3\textwidth}
	\vspace{-\baselineskip} % Required for vertically aligning minipages
	
\cvsect{Persönliches und Hobbys}
	
	Ich bin kürzlich (2024) Vater geworden und verbringe sehr gerne Zeit mit meiner Familie. Außerdem gehe ich gerne wandern und klettern.
\end{minipage}
\hfill
\mbox{}
\vfill
\begin{minipage}[t]{1.0\textwidth}
\centering{\footnotesize{Dieses Dokument wurde \generated~am \today erstellt. Quellcode verfügbar auf \color{blue}{\href{https://github.com/mxlnk/cv}{GitHub}}.}}

\end{minipage}

%----------------------------------------------------------------------------------------

\end{document}
